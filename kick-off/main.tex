%----------------------------------------------------------------------------------------
%    PACKAGES AND THEMES
%----------------------------------------------------------------------------------------

\documentclass[aspectratio=169,xcolor=dvipsnames]{beamer}
\usetheme{SimplePlus}

\usepackage{hyperref}
\usepackage{graphicx} % Allows including images
\usepackage{booktabs} % Allows the use of \toprule, \midrule and \bottomrule in tables

%----------------------------------------------------------------------------------------
%    TITLE PAGE
%----------------------------------------------------------------------------------------

\title{AR Flood Hazard Maps}
% \subtitle{Subtitle}

\author{Frederik Alpers, Lea Plümacher, Marvin Hagemeister}

\institute
{
    Freie Universität Berlin % Your institution for the title page
}
\date{\today} % Date, can be changed to a custom date

%----------------------------------------------------------------------------------------
%    PRESENTATION SLIDES
%----------------------------------------------------------------------------------------

\begin{document}

\begin{frame}
    % Print the title page as the first slide
    \titlepage
\end{frame}

\begin{frame}{Overview}
    % Throughout your presentation, if you choose to use \section{} and \subsection{} commands, these will automatically be printed on this slide as an overview of your presentation
    \tableofcontents
\end{frame}

%------------------------------------------------
\section{Team}
%------------------------------------------------

\begin{frame}{Team}
    \begin{itemize}
        \item Frederik Alpers 
        \item Lea Plümacher
        \item Marvin Hagemeister (Master Informatik)
    \end{itemize}
\end{frame}

%------------------------------------------------
\section{Project Idea}
%------------------------------------------------

\begin{frame}{Project Idea}
    \centering
    \textbf{Mit Hilfe von Augmented Reality potentielle Überflutungen und Meeresspiegelanstiege intuitiv verstehbar machen und somit einen besseren Schutz der Bevölkerung erreichen.}
\end{frame}

\begin{frame}{Project Idea}
    \begin{columns}[c] % The "c" option specifies centered vertical alignment while the "t" option is used for top vertical alignment

        \column{.45\textwidth} % Left column and width
        \begin{enumerate}
            \item AR Handy App
            \item User Location per GPS
            \item Anzeigen von historischen Daten und Vorhersagen
        \end{enumerate}

        \column{.45\textwidth} % Right column and width
        \begin{figure}
          \includegraphics[width=\linewidth]{img/Screenshot 2025-10-26 at 12-46-46 02_SWP_ZLAB_Project_Drafts.pdf.png}
          \caption{Concept of the app}
          \label{fig:boat1}
        \end{figure}

    \end{columns}
\end{frame}

%------------------------------------------------
\section{Data Sources}
%------------------------------------------------

\begin{frame}{Project Idea}
    \begin{columns}[c] % The "c" option specifies centered vertical alignment while the "t" option is used for top vertical alignment

        \column{.45\textwidth} % Left column and width
        \begin{enumerate}
            \item Nur Berlin: https://wasserportal.berlin.de
            \item REST-API von Pegelonline: https://www.pegelonline.wsv.de/
            \item Länderübergreifendes Hochwasser Portal (LHP): https://www.hochwasserzentralen.de/
        \end{enumerate}

        \column{.45\textwidth} % Right column and width
        \begin{figure}
          \includegraphics[width=\linewidth]{img/Screenshot 2025-10-26 at 13-36-55 Hochwasser Aktuelle Situation und Warnungen LHP.png}
          \caption{Startseite von LHP, stellt auch APIs bereit.}
          \label{fig:pegelonline}
        \end{figure}

    \end{columns}
\end{frame}

%------------------------------------------------
\section{Suggested Medium}
%------------------------------------------------

\begin{frame}{Suggested Medium}
    \begin{enumerate}
        \item Smartphones / Web-AR
        \item Zugänglich für die Meisten
        \item Gut verwendbar für Location based APPs
    \end{enumerate}
\end{frame}

%------------------------------------------------
\section{Prototype / Mock-up}
%------------------------------------------------

\begin{frame}{Prototype / Mock-up}
    \begin{figure}
          \includegraphics[width=0.6\linewidth]{img/photo_5472335260990701058_y.jpg}
          \caption{Erstes Mock-up für die AR Flood Hazard Map App.}
          \label{fig:pegelonline}
        \end{figure}
\end{frame}

%------------------------------------------------
\section{Mile stones}
%------------------------------------------------

\begin{frame}{Mile Stones}
    \textbf{1. Recherche und endgültiges Konzept}
    \begin{itemize}
        \item Projektidee genau definieren
        \item Detaillierteres Mock-up, evtl. User Stories
        \item APIs und Datenquellen prüfen
        \item Zeithorizont: 2 Wochen (10.11)
    \end{itemize}
\end{frame}

\begin{frame}{Mile Stones}
    \textbf{2. AR-Prototyp}
    \begin{itemize}
        \item Erste Web / AR Tests
        \item GPS, Gyroskop und einfaches Overlay
        \item Fake Pegelstände
        \item Verwenden der Kamera
        \item Zeithorizont: 01.12
    \end{itemize}
\end{frame}

\begin{frame}{Mile Stones}
    \textbf{3. Pegelstandssimulation}
    \begin{itemize}
        \item Darstellung realer Pegelstände
        \item Zeithorizont: 2 Wochen (15.12)
    \end{itemize}
\end{frame}

\begin{frame}{Mile Stones}
    \textbf{4. UX \& Testen}
    \begin{itemize}
        \item Hinzufügen von Animationen
        \item Zeitstrahl und andere UI Elemente
        \item Zeithorizont: 2 Wochen (12.01)
    \end{itemize}
\end{frame}

\begin{frame}{Mile Stones}
    \textbf{5. Präsentation}
    \begin{itemize}
        \item Testen und Bug-Fixes
        \item Dokumentation
        \item Zeithorizont: 2 Wochen (29.01.2026)
    \end{itemize}
\end{frame}

\begin{frame}
    \Huge{\centerline{\textbf{Fragen?}}}
\end{frame}

%----------------------------------------------------------------------------------------

\end{document}
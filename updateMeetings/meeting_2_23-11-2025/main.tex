%----------------------------------------------------------------------------------------
%    PACKAGES AND THEMES
%----------------------------------------------------------------------------------------

\documentclass[aspectratio=169,xcolor=dvipsnames]{beamer}
\usetheme{SimplePlus}

\usepackage{hyperref}
\usepackage{graphicx} % Allows including images
\usepackage{booktabs} % Allows the use of \toprule, \midrule and \bottomrule in tables
\usepackage{lmodern}
\usepackage{pifont} % Für Checkmarks

% Definition von Checkmarks und Crossmarks
\newcommand{\cmark}{\textcolor{green!60!black}{\ding{51}}}
\newcommand{\xmark}{\textcolor{red}{\ding{55}}}
\newcommand{\omark}{\textcolor{orange}{\ding{110}}} % Offen/In Arbeit

%----------------------------------------------------------------------------------------
%    TITLE PAGE
%----------------------------------------------------------------------------------------

\title{AR Flood Hazard Maps}
\subtitle{Update Meeting 2}

\author{Frederik Alpers, Lea Plümacher, Marvin Hagemeister}

\institute
{
    Freie Universität Berlin % Your institution for the title page
}
\date{\today} % Date, can be changed to a custom date

%----------------------------------------------------------------------------------------
%    PRESENTATION SLIDES
%----------------------------------------------------------------------------------------

\begin{document}

\begin{frame}
    % Print the title page as the first slide
    \titlepage
\end{frame}

\begin{frame}{Overview}
    % Throughout your presentation, if you choose to use \section{} and \subsection{} commands, these will automatically be printed on this slide as an overview of your presentation
    \tableofcontents
\end{frame}

% --- Sektion 1: Aktueller Sprint Status ---
\section{Sprint Status}

\begin{frame}{Sprint Ziele \& Status}
    Wir befinden uns aktuell in der Mitte des Sprints (bis 1.12).
    \vspace{0.5cm}
    \begin{itemize}
        \item[\cmark] \textbf{Erster AR-Prototyp}
        \item[\cmark] \textbf{Deployment:} App auf Gerät geflasht
        \item[\cmark] \textbf{GPS:} Position (Längengrad, Breitengrad und Höhe) bestimmt
        \item[\omark] \textbf{Simulation:} Fake-API für Wasserstände (Geplant bis Ende der Woche)
    \end{itemize}
\end{frame}

% --- Sektion 2: Detaillierte Ergebnisse (Bild + Text) ---
\section{Erreichte Meilensteine}

\begin{frame}{Erster AR-Prototyp}
    \begin{columns}[C] % T = Top aligned
        \begin{column}{0.48\textwidth}
            \textbf{Funktionalität:}
            \begin{itemize}
                \item Example AR-Unity Projekt erstellt / erweitert
                \item Erkennung von Flächen funktioniert 
                \item Funktioniert auf unseren Testgeräten
            \end{itemize}
        \end{column}
        
        \begin{column}{0.48\textwidth}
            \centering
            % HIER BILD EINFÜGEN: \includegraphics[width=\textwidth]{ar_screenshot.png}
            \framebox{\parbox{0.9\textwidth}{\centering
                \vspace{2.5cm}
                \textbf{Screenshot: AR Prototyp} \\
                \small\textit{Screenshot von Unity mit AR-Example-Project}
                \vspace{2.5cm}
            }}
        \end{column}
    \end{columns}
\end{frame}

\begin{frame}{GPS Location \& Hight}
    \begin{columns}[C]
        \begin{column}{0.48\textwidth}
            \textbf{Implementierung:}
            \begin{itemize}
                \item Zugriff auf Android Location Services
                \item Anzeige von Breitengrad, Längengrad, Genauigkeit und Höhe
            \end{itemize}
            \vspace{0.5cm}
            \textit{Grundlage für das Finden der richtigen Messtation und Berechnung des Wasserstandes.}
        \end{column}
        
        \begin{column}{0.48\textwidth}
            \centering
            \includegraphics[width=0.5\textwidth]{./img/prot_1_location.jpg}
%            \framebox{\parbox{0.9\textwidth}{\centering
%                \vspace{2.5cm}
%                \textbf{Screenshot: GPS Daten} \\
%                \small\textit{UI mit Koordinaten-Anzeige}
%                \vspace{2.5cm}
%            }}
        \end{column}
    \end{columns}
\end{frame}

% --- Sektion 3: Restlicher Sprint ---
\section{Nächste Schritte im Sprint}

\begin{frame}{Planung bis Sprint-Ende (Nächste Woche)}
    Ein offener Punkt für diesen Sprint:
    \vspace{0.5cm}
    
    \begin{block}{To-Do: Fake-API Implementierung}
        Um die Wasseroberfläche simulieren zu können, bevor die echte API angebunden ist, benötigen wir statische Daten.
    \end{block}
    
    \textbf{Geplante Umsetzung:}
    \begin{itemize}
        \item Erstellung eines Mock-Services
        \item Rückgabe von fixen Pegelständen für Testzwecke
        \item Ermöglicht einfaches Testing positionsunabhängig
    \end{itemize}
\end{frame}

% --- Sektion 4: Projekt Definitionen ---
\section{Projekt Definitionen}

\begin{frame}{Projekt-Anforderungen}
    Wir haben die Anforderungen neu definiert, um den Fokus für das MVP zu wahren.
    
    \begin{alertblock}{Minimal Requirements (MVP)}
        \begin{itemize}
            \item Lauffähige AR App (Android)
            \item Anzeige eines Flut-Levels (visuell als Plane)
            \item Interface zur Anzeige der Wasserstandshöhe (als Textobjekt)
            \item Mock-API (selbes Format wie Real API) für Testing und Demonstration
        \end{itemize}
    \end{alertblock}
\end{frame}

\begin{frame}{Nice-to-Have Features}
    Ziele für spätere Ausbaustufen:
    
    \begin{exampleblock}{Erweiterungen}
        \begin{itemize}
            \item \textbf{Daten:} Echte API-Anbindung \& Live GPS-Abgleich
            \item \textbf{Feature:} Vorhersagen/Prognosen per Zeitstrahl auswählen
            \item \textbf{Interaktion:} Standort-Teleportation \& manuelle Eingabe (Slider)
            \item \textbf{Visualisierung:} 
            \begin{itemize}
                \item Realistische Wasserphysik
                \item Rendering unter der Wasseroberfläche
                \item Referenzobjekte (Autos, Schilder, Enten) zum Größenvergleich
            \end{itemize}
        \end{itemize}
    \end{exampleblock}
\end{frame}

% --- Sektion 5: Projektmanagement ---
\section{Organisation}

\begin{frame}{Projektmanagement \& Roadmap}
    \begin{columns}[T]
        \begin{column}{0.48\textwidth}
            \textbf{Strukturierung:}
            \begin{itemize}
                \item Alle Anforderungen wurden in GitHub Issues überführt.
                \item Roadmap wurde visualisiert, um Abhängigkeiten zu klären.
            \end{itemize}
        \end{column}
        
        \begin{column}{0.48\textwidth}
            \centering
            % HIER BILD EINFÜGEN: \includegraphics[width=\textwidth]{github_roadmap.png}
            \framebox{\parbox{0.9\textwidth}{\centering
                \vspace{1.5cm}
                \textbf{Screenshot: GitHub/Roadmap} \\
                \small\textit{Übersicht der Issues \& Zeitplan}
                \vspace{1.5cm}
            }}
        \end{column}
    \end{columns}
\end{frame}

% --- Sektion 6: Ausblick ---
\section{Ausblick}

\begin{frame}{Sneak Peak: Der nächste Sprint}
    \begin{center}
        \Large \textbf{Ziel bis 15. Dezember}
    \end{center}
    \vspace{0.5cm}
    
    Unsere Ziele für den kommenden Sprint:
    
    \begin{itemize}
        \item \textbf{Integration der Mock-API:} Abruf realer Pegeldaten basierend auf Geolocation.
        \item \textbf{Visualisierung:} Anzeige der Wasserstände als Plane und in Textfeldern.
    \end{itemize}
\end{frame}

\begin{frame}
    \Huge{\centerline{\textbf{Fragen?}}}
\end{frame}

%----------------------------------------------------------------------------------------

\end{document}

%----------------------------------------------------------------------------------------
%    PACKAGES AND THEMES
%----------------------------------------------------------------------------------------

\documentclass[aspectratio=169,xcolor=dvipsnames]{beamer}
\usetheme{SimplePlus}

\usepackage{hyperref}
\usepackage{graphicx} % Allows including images
\usepackage{booktabs} % Allows the use of \toprule, \midrule and \bottomrule in tables
%\usepackage{lmodern}
\usepackage{pifont} % Für Checkmarks
\usepackage{fontawesome5}

%\usepackage{media9}
\usepackage{pdfpc}


% Definition von Checkmarks und Crossmarks
\newcommand{\cmark}{\textcolor{green!60!black}{\ding{51}}}
\newcommand{\xmark}{\textcolor{red}{\ding{55}}}
\newcommand{\omark}{\textcolor{orange}{\ding{110}}} % Offen/In Arbeit

% TikZ für Diagramme
\usepackage{tikz}
\usetikzlibrary{shapes.geometric, arrows.meta, positioning, shadows, calc, fit, backgrounds}

%----------------------------------------------------------------------------------------
%    TITLE PAGE
%----------------------------------------------------------------------------------------

\title{AR Flood Hazard Maps}
\subtitle{Update Meeting 4}

\author{Frederik Alpers, Lea Plümacher, Marvin Hagemeister}

\institute
{
    Freie Universität Berlin % Your institution for the title page
}
\date{\today} % Date, can be changed to a custom date

%----------------------------------------------------------------------------------------
%    PRESENTATION SLIDES
%----------------------------------------------------------------------------------------

\begin{document}

\begin{frame}
    % Print the title page as the first slide
    \titlepage
\end{frame}

% \begin{frame}{Overview}
%     % Throughout your presentation, if you choose to use \section{} and \subsection{} commands, these will automatically be printed on this slide as an overview of your presentation
%     \tableofcontents
% \end{frame}

%%%%%
% Progress Presentation: Notwendige Punkte:
% - Project motivation
% - Progress on your project idea - Where are you currently at? What comes next? (timeline / deviation from original time line) 
% - Current implementation status – Include screenshots or videos if possible
% - Reflection – How is it going so far? What have you learned? What do you want to learn in the next weeks.
% - Feedback (optional) – What can we do to improve your experience?

% --- Sektion 1: Project motivation
\section{Project Motivation}


\begin{frame}{Project Motivation}
  \begin{columns}[c]
    \begin{column}{0.48\textwidth}
      \textbf{Bestehende Informationsmöglichkeiten beschränkt durch:}
      \begin{itemize}
        \item Abstrakte Pegelwerte ohne Bezug zur eigenen Umgebung
        \item Lokale Geländeunterschiede werden nicht berücksichtigt
        \item Fehlende visuelle Einschätzung des persönlichen Risikos
      \end{itemize}
    \end{column}
    
    \begin{column}{0.48\textwidth}
      \centering
      \includegraphics[width=0.8\textwidth]{./img/hwz.png}

      \vspace{0.5em}
      {\footnotesize
      Screenshot: Aktuelle Hochwasserlage Deutschland  
      (Quelle: \texttt{hochwasserzentralen.de})}
    \end{column}

  \end{columns}
\end{frame}

\begin{frame}{Project Idea}
  \begin{columns}[c]
    \begin{column}{0.48\textwidth}
      \textbf{Die Anwendung soll unter anderem folgende Punkte beinhalten:}
      \begin{itemize}
        \item AR Handy App
        \item Interaktive Anzeige des Wasserstandes vor Ort
        \item Anzeigen von aktuellen Daten und Vorhersagen
      \end{itemize}
    \end{column}

    \begin{column}{0.48\textwidth}
      \centering
      \includegraphics[width=0.8\textwidth]{./img/mock_up.jpg}

      \vspace{0.5em}
      {\footnotesize
      Mock-Up}
    \end{column}

    % \begin{column}{0.48\textwidth}
    %   \centering
    %   \framebox{\parbox{0.9\textwidth}{\centering
    %       \vspace{2.5cm}
    %       \textbf{Screenshot: AR Prototyp} \\
    %       \small\textit{Screenshot von Unity, Zeichnung oder AR-Example-Project}
    %       \vspace{2.5cm}
    %   }}

    %   %      \vspace{0.5em}
    %   %      {\footnotesize
    %   %      Screenshot: Aktuelle Hochwasserlage Deutschland  
    %   %      (Quelle: \texttt{hochwasserzentralen.de})}
    % \end{column}

  \end{columns}

  % Gibt es Abweichungen zu unserer ursprünglichen Projektidee?
  % Was wird unser Projekt am Ende voraussichtlich beinhalten?
\end{frame}


% --- Sektion 2: Progress on your project idea - Where are you currently at? What comes next? 
\section{Progress}

% \begin{frame}{Project Requirements}
%   \begin{exampleblock}{Requirements}
%     \begin{itemize}
%       \item Lauffähige AR App (Android)
%       \item Anzeige eines Flut-Levels (visuell als Plane)
%       \item Interface zur Anzeige der Wasserstandshöhe (als Textobjekt)
%       \item Mock-API (selbes Format wie Real API) für Testing und Demonstration
%       \item Übergang von Mock-API zu echter API
%       \item Verbesserung der Grafik und UI Elemente
%       \item Dokumentation  
%     \end{itemize}
%   \end{exampleblock}
% \end{frame}

\begin{frame}{Project Requirements}
  \begin{alertblock}{minimal Requirements}
    \begin{itemize}
      \item %Lauffähige AR App (Android)
      \item %Anzeige eines Flut-Levels (visuell als Plane)
      \item %Interface zur Anzeige der Wasserstandshöhe (als Textobjekt)
      \item %Mock-API (selbes Format wie Real API) für Testing und Demonstration  
    \end{itemize}
  \end{alertblock}

    \begin{block}{additional Requirements}
    \begin{itemize}
      \item %Übergang von Mock-API zu echter API
      \item  %Verbesserung der Grafik und UI Elemente
      \item %Dokumentation  
      \item 
    \end{itemize}
  \end{block}
\end{frame}
\begin{frame}{Project Requirements}
  \begin{alertblock}{minimal Requirements}
    \begin{itemize}
      \item Lauffähige AR App (Android)
      \item %Anzeige eines Flut-Levels (visuell als Plane)
      \item %Interface zur Anzeige der Wasserstandshöhe (als Textobjekt)
      \item %Mock-API (selbes Format wie Real API) für Testing und Demonstration  
    \end{itemize}
  \end{alertblock}

    \begin{block}{additional Requirements}
    \begin{itemize}
      \item %Übergang von Mock-API zu echter API
      \item  %Verbesserung der Grafik und UI Elemente
      \item %Dokumentation 
      \item  
    \end{itemize}
  \end{block}
\end{frame}

\begin{frame}{Project Requirements}
  \begin{alertblock}{minimal Requirements}
    \begin{itemize}
      \item Lauffähige AR App (Android)
      \item Anzeige eines Flut-Levels (visuell als Plane)
      \item %Interface zur Anzeige der Wasserstandshöhe (als Textobjekt)
      \item %Mock-API (selbes Format wie Real API) für Testing und Demonstration  
    \end{itemize}
  \end{alertblock}

    \begin{block}{additional Requirements}
    \begin{itemize}
      \item %Übergang von Mock-API zu echter API
      \item  %Verbesserung der Grafik und UI Elemente
      \item %Dokumentation  
      \item 
    \end{itemize}
  \end{block}
\end{frame}

\begin{frame}{Project Requirements}
  \begin{alertblock}{minimal Requirements}
    \begin{itemize}
      \item Lauffähige AR App (Android)
      \item Anzeige eines Flut-Levels (visuell als Plane)
      \item Interface zur Anzeige der Wasserstandshöhe (als Textobjekt)
      \item %Mock-API (selbes Format wie Real API) für Testing und Demonstration  
    \end{itemize}
  \end{alertblock}

    \begin{block}{additional Requirements}
    \begin{itemize}
      \item %Übergang von Mock-API zu echter API
      \item  %Verbesserung der Grafik und UI Elemente
      \item %Dokumentation  
      \item 
    \end{itemize}
  \end{block}
\end{frame}

\begin{frame}{Project Requirements}
  \begin{alertblock}{minimal Requirements}
    \begin{itemize}
      \item Lauffähige AR App (Android)
      \item Anzeige eines Flut-Levels (visuell als Plane)
      \item Interface zur Anzeige der Wasserstandshöhe (als Textobjekt)
      \item Mock-API (selbes Format wie Real API) für Testing und Demonstration  
    \end{itemize}
  \end{alertblock}

    \begin{block}{additional Requirements}
    \begin{itemize}
      \item %Übergang von Mock-API zu echter API
      \item  %Verbesserung der Grafik und UI Elemente
      \item %Dokumentation  
      \item 
    \end{itemize}
  \end{block}
\end{frame}

\begin{frame}{Implementation Status | Screenshot}

    \begin{center}
      \centering
      \includegraphics[height=0.75\textheight,keepaspectratio]{./img/screenshot_aktuellesProjekt.png}

      \vspace{0.4em}
      {\footnotesize minimal viable product}
    \end{center}

\end{frame}

\begin{frame}{Project Requirements}
  \begin{alertblock}{minimal Requirements}
    \begin{itemize}
      \item[\cmark] Lauffähige AR App (Android)
      \item[\cmark] Anzeige eines Flut-Levels (visuell als Plane)
      \item[\cmark] Interface zur Anzeige der Wasserstandshöhe (als Textobjekt)
      \item[\cmark] Mock-API (selbes Format wie Real API) für Testing und Demonstration  
    \end{itemize}
  \end{alertblock}

    \begin{block}{additional Requirements}
    \begin{itemize}
      \item %Übergang von Mock-API zu echter API
      \item  %Verbesserung der Grafik und UI Elemente
      \item %Dokumentation  
      \item 
    \end{itemize}
  \end{block}
\end{frame}

\begin{frame}{Project Requirements}
  \begin{alertblock}{minimal Requirements}
    \begin{itemize}
      \item[\cmark] Lauffähige AR App (Android)
      \item[\cmark] Anzeige eines Flut-Levels (visuell als Plane)
      \item[\cmark] Interface zur Anzeige der Wasserstandshöhe (als Textobjekt)
      \item[\cmark] Mock-API (selbes Format wie Real API) für Testing und Demonstration  
    \end{itemize}
  \end{alertblock}

    \begin{block}{additional Requirements}
    \begin{itemize}
      \item Übergang von Mock-API zu echter API
      \item  %Verbesserung der Grafik und UI Elemente
      \item %Dokumentation  
      \item 
    \end{itemize}
  \end{block}
\end{frame}

\begin{frame}{Project Requirements}
  \begin{alertblock}{minimal Requirements}
    \begin{itemize}
      \item[\cmark] Lauffähige AR App (Android)
      \item[\cmark] Anzeige eines Flut-Levels (visuell als Plane)
      \item[\cmark] Interface zur Anzeige der Wasserstandshöhe (als Textobjekt)
      \item[\cmark] Mock-API (selbes Format wie Real API) für Testing und Demonstration  
    \end{itemize}
  \end{alertblock}

    \begin{block}{additional Requirements}
    \begin{itemize}
      \item Übergang von Mock-API zu echter API
      \item Verbesserung der Grafik und UI Elemente
      \item %Dokumentation  
      \item 
    \end{itemize}
  \end{block}
\end{frame}

\begin{frame}{Project Requirements}
  \begin{alertblock}{minimal Requirements}
    \begin{itemize}
      \item [\cmark]Lauffähige AR App (Android)
      \item [\cmark]Anzeige eines Flut-Levels (visuell als Plane)
      \item [\cmark]Interface zur Anzeige der Wasserstandshöhe (als Textobjekt)
      \item [\cmark]Mock-API (selbes Format wie Real API) für Testing und Demonstration  
    \end{itemize}
  \end{alertblock}

    \begin{block}{additional Requirements}
    \begin{itemize}
      \item Übergang von Mock-API zu echter API
      \item Verbesserung der Grafik und UI Elemente
      \item Dokumentation  
      \item 
    \end{itemize}
  \end{block}
\end{frame}

\begin{frame}{Project Requirements}
  \begin{alertblock}{minimal Requirements}
    \begin{itemize}
      \item[\cmark] Lauffähige AR App (Android)
      \item[\cmark] Anzeige eines Flut-Levels (visuell als Plane)
      \item[\cmark] Interface zur Anzeige der Wasserstandshöhe (als Textobjekt)
      \item[\cmark] Mock-API (selbes Format wie Real API) für Testing und Demonstration  
    \end{itemize}
  \end{alertblock}

    \begin{block}{additional Requirements}
    \begin{itemize}
      \item Übergang von Mock-API zu echter API
      \item Verbesserung der Grafik und UI Elemente
      \item Dokumentation 
      \item Zeitstrahl 
    \end{itemize}
  \end{block}
\end{frame}

\begin{frame}{Current Implementation Status | Screenshot 2}

    \begin{center}
      \centering
      \includegraphics[height=0.75\textheight,keepaspectratio]{./img/final_screenshot.jpg}

      \vspace{0.4em}
      {\footnotesize Finale Implementierung mit simuliertem Wasserstand}
    \end{center}

\end{frame}

\begin{frame}{Current Implementation Status | Screenshot 2}

    \begin{center}
      \centering
      \includegraphics[height=0.75\textheight,keepaspectratio, trim={0 35cm 0 0},clip]{./img/final_screenshot.jpg}

      \vspace{0.4em}
      {\footnotesize Finale Implementierung mit simuliertem Wasserstand}
    \end{center}

\end{frame}


\begin{frame}{Project Requirements}
  \begin{alertblock}{minimal Requirements}
    \begin{itemize}
      \item[\cmark] Lauffähige AR App (Android)
      \item[\cmark] Anzeige eines Flut-Levels (visuell als Plane)
      \item[\cmark] Interface zur Anzeige der Wasserstandshöhe (als Textobjekt)
      \item[\cmark] Mock-API (selbes Format wie Real API) für Testing und Demonstration  
    \end{itemize}
  \end{alertblock}

    \begin{block}{additional Requirements}
    \begin{itemize}
      \item[\cmark] Übergang von Mock-API zu echter API
      \item[\textcolor{orange}{\ding{109}}] Verbesserung der Grafik und UI Elemente
      \item[\textcolor{orange}{\ding{109}}] Dokumentation  
      \item[\textcolor{gray}{\ding{109}}] Zeitstrahl  
    \end{itemize}
  \end{block}
\end{frame}



\begin{frame}{Progress | Timeline}

    \includegraphics[height=0.73\textheight,keepaspectratio]{./img/progress.png}

\end{frame}

\section{Implementation}

\begin{frame}{Technical implementation | Software Used}

    \begin{center}
      \centering
      \includegraphics[height=0.75\textheight,keepaspectratio]{./img/screenshot_unity_neu.png}

      \vspace{0.4em}
      {\footnotesize Screenshot of Unity}
    \end{center}

\end{frame}

\begin{frame}{Technical Implementation | Diagram}
  \begin{tikzpicture}[
    api/.style={
      draw,
      rounded corners,
      minimum width=4cm,
      minimum height=1.0cm,
      align=center,
      path picture={
        \draw[dashed]
          ([yshift=0mm]path picture bounding box.center -|
          path picture bounding box.west)
          --
          ([yshift=0mm]path picture bounding box.center -|
          path picture bounding box.east);
      }
    },
    app/.style={
      draw,
      rounded corners,
      minimum width=3cm,
      minimum height=2cm,
      align=center
    },
    decision/.style={
      circle,
      draw,
      minimum size=10mm,
      align=center,
      font=\small
    },
    node distance=0.4cm and 2.8cm,
    >={Stealth}
    ]

    % Left column APIs
    \node[api] (stations) {Pegelonline API\\[2mm]Fetch all stations};
    \node[api, below=of stations] (levels) {Pegelonline API\\[2mm]Fetch water levels};
    \node[api, below=of levels] (mock) {Mock API\\[2mm]Testing};
    \node[api, below=of mock] (elevation) {Open--Meteo\\[2mm]Elevation API};

    % Decision / junction
    \node[decision, right=3cm of levels] (decision) {Mock\\API?};

    % Client app
    \node[app, right=2.5cm of decision, yshift=-16mm] (client) {Client\\App};

    % GPS
    \node[api, above=1.5cm of client, minimum width=3cm] (gps)
      {GPS\\[2mm]Location};

    % Connections
    \draw[->] (stations.east) -- node[above, font=\small] {no} (decision.west);
    \draw[->] (levels.east) -- (decision.west);
    \draw[->, dashed] (mock.east) -- node[below, font=\small] {yes} (decision.west);

    \draw[<->] (decision.east) -- (client.west);
    \draw[->] (gps.south) -- (client.north);

    \draw[->]
      (elevation.east)
      .. controls +(2,0) and +(-2,-1) ..
      (client.south);

  \end{tikzpicture}

\end{frame}


\begin{frame}{Future Works | Technical Enhancements}
  \begin{columns}[c]
    \begin{column}{0.48\textwidth}
      \textbf{Predictive Flood Modeling}
      \begin{itemize}
        \item Integration von Wettervorhersagen
        \item Zeitbasierte Animation des Pegelanstiegs/-rückgangs
        \item Frühwarnsystem bei kritischen Schwellenwerten
      \end{itemize}
      
      \vspace{1em}
      \textbf{Erwarteter Nutzen:}
      \begin{itemize}
        \item Bessere Vorbereitung für Anwohner
        \item Rechtzeitige Evakuierung möglich
        \item Visualisierung von möglichen Szenarien
      \end{itemize}
    \end{column}

    \begin{column}{0.48\textwidth}
      \centering
      \begin{tikzpicture}[scale=0.8]

        \draw[->] (0,0) -- (5,0) node[right] {Zeit};
        \draw[->] (0,0) -- (0,4) node[above] {Pegel};
        
        \draw[dashed, gray] (1.5,0) -- (1.5,4);
        \node[below] at (1.5,0) {\small Jetzt};
        
        \draw[thick, blue] (0,1.5) -- (1.5,2);
        
        \draw[thick, red, dashed] (1.5,2) -- (3,2.8) -- (4,3.2);
        \draw[thick, orange, dashed] (1.5,2) -- (3,2.3) -- (4,2.5);
        \draw[thick, green, dashed] (1.5,2) -- (3,1.8) -- (4,1.6);
        
        \draw[dotted, thick] (0,3) -- (5,3);
        \node[right, text=red] at (5,3) {\tiny Kritisch};
      \end{tikzpicture}
      
      \vspace{0.5em}
      {\tiny
      \textcolor{blue}{—} aktuell \quad
      \textcolor{red}{- -} Worst Case \quad
      \textcolor{orange}{- -} Wahrscheinlich \quad
      \textcolor{green}{- -} Best Case
      }
      
      \vspace{0.5em}
      {\footnotesize Konzept: Vorhersagemodell mit verschiedenen Szenarien}
    \end{column}
  \end{columns}
\end{frame}

\begin{frame}{Future Works | Additional Enhancements}
  \begin{columns}[T]
    \begin{column}{0.48\textwidth}
      
      \vspace{0.5em}
      
      \textbf{Offline-Modus}
      \begin{itemize}
        \item Daten-Caching für Gebiete ohne Netz
        \item Lokale Datenspeicherung
        \item Synchronisation bei Verbindung
      \end{itemize}

      \vspace{0.5em}
      
      \textbf{Evakuierungsrouten}
      \begin{itemize}
        \item Integration mit Navigation
        \item Sichere Wege zu Sammelstellen
        \item Echtzeit-Aktualisierung bei Überflutung
      \end{itemize}
      
    \end{column}

    \begin{column}{0.48\textwidth}

      \vspace{0.5em}
      \textbf{Optimierung}
      \begin{itemize}
        \item Performance-Optimierung (Akku, Rendering)
        \item Wasser-Animation (Richtung, Geschwindigkeit)
        \item auf mehr Geräten testen 
      \end{itemize}
      
      \end{column}
  \end{columns}
\end{frame}


\begin{frame}{Reflection}
  \begin{itemize}
      \item Läuft bisher gut
      \item Mock-API gute Entscheidung, um schnell zu testen
      \item Unity ist möglicherweise zu aufwendig für Anwendungsfall
  \end{itemize}
\end{frame}

\begin{frame}{Reflection | Future Learnings}
  \begin{itemize}
    \item[\faPlus] Was macht eine intuitive und simple UI aus?
    \item[\faPlus] AR Darstellung von Wetterdaten einfacher zu verstehen?
  \end{itemize}
\end{frame}



\begin{frame}{Organisation}
\begin{itemize}
    \item Treffen Montags
    \item Textchat
    \item GitHub
    \item Unity Cloud
\end{itemize}
\end{frame}

\begin{frame}
    \Huge{\centerline{\textbf{Fragen?}}}
\end{frame}






%----------------------------------------------------------------------------------------

\end{document}

%----------------------------------------------------------------------------------------
%    PACKAGES AND THEMES
%----------------------------------------------------------------------------------------

\documentclass[aspectratio=169,xcolor=dvipsnames]{beamer}
\usetheme{SimplePlus}

\usepackage{hyperref}
\usepackage{graphicx} % Allows including images
\usepackage{booktabs} % Allows the use of \toprule, \midrule and \bottomrule in tables
%\usepackage{lmodern}
\usepackage{pifont} % Für Checkmarks
\usepackage{fontawesome5}

% Definition von Checkmarks und Crossmarks
\newcommand{\cmark}{\textcolor{green!60!black}{\ding{51}}}
\newcommand{\xmark}{\textcolor{red}{\ding{55}}}
\newcommand{\omark}{\textcolor{orange}{\ding{110}}} % Offen/In Arbeit

%----------------------------------------------------------------------------------------
%    TITLE PAGE
%----------------------------------------------------------------------------------------

\title{AR Flood Hazard Maps}
\subtitle{Update Meeting 4}

\author{Frederik Alpers, Lea Plümacher, Marvin Hagemeister}

\institute
{
    Freie Universität Berlin % Your institution for the title page
}
\date{\today} % Date, can be changed to a custom date

%----------------------------------------------------------------------------------------
%    PRESENTATION SLIDES
%----------------------------------------------------------------------------------------

\begin{document}

\begin{frame}
    % Print the title page as the first slide
    \titlepage
\end{frame}

% \begin{frame}{Overview}
%     % Throughout your presentation, if you choose to use \section{} and \subsection{} commands, these will automatically be printed on this slide as an overview of your presentation
%     \tableofcontents
% \end{frame}

%%%%%
% Progress Presentation: Notwendige Punkte:
% - Project motivation
% - Progress on your project idea - Where are you currently at? What comes next? (timeline / deviation from original time line) 
% - Current implementation status – Include screenshots or videos if possible
% - Reflection – How is it going so far? What have you learned? What do you want to learn in the next weeks.
% - Feedback (optional) – What can we do to improve your experience?

% --- Sektion 1: Project motivation
\section{Project Motivation}


\begin{frame}{Project Motivation}
  \begin{columns}[c]
    \begin{column}{0.48\textwidth}
      \textbf{Bestehende Informationsmöglichkeiten beschränkt durch:}
      \begin{itemize}
        \item Abstrakte Pegelwerte ohne Bezug zur eigenen Umgebung
        \item Lokale Geländeunterschiede werden nicht berücksichtigt
        \item Fehlende visuelle Einschätzung des persönlichen Risikos
      \end{itemize}
    \end{column}
    
    \begin{column}{0.48\textwidth}
      \centering
      \includegraphics[width=0.8\textwidth]{./img/hwz.png}

      \vspace{0.5em}
      {\footnotesize
      Screenshot: Aktuelle Hochwasserlage Deutschland  
      (Quelle: \texttt{hochwasserzentralen.de})}
    \end{column}

  \end{columns}
\end{frame}

\begin{frame}{Project Idea}
  \begin{columns}[c]
    \begin{column}{0.48\textwidth}
      \textbf{Die Anwendung soll unter anderem folgende Punkte beinhalten:}
      \begin{itemize}
        \item AR Handy App
        \item Interaktive Anzeige des Wasserstandes vor Ort
        \item Anzeigen von aktuellen Daten und Vorhersagen
      \end{itemize}
    \end{column}

    \begin{column}{0.48\textwidth}
      \centering
      \includegraphics[width=0.8\textwidth]{./img/mock_up.jpg}

      \vspace{0.5em}
      {\footnotesize
      Mock-Up}
    \end{column}

    % \begin{column}{0.48\textwidth}
    %   \centering
    %   \framebox{\parbox{0.9\textwidth}{\centering
    %       \vspace{2.5cm}
    %       \textbf{Screenshot: AR Prototyp} \\
    %       \small\textit{Screenshot von Unity, Zeichnung oder AR-Example-Project}
    %       \vspace{2.5cm}
    %   }}

    %   %      \vspace{0.5em}
    %   %      {\footnotesize
    %   %      Screenshot: Aktuelle Hochwasserlage Deutschland  
    %   %      (Quelle: \texttt{hochwasserzentralen.de})}
    % \end{column}

  \end{columns}

  % Gibt es Abweichungen zu unserer ursprünglichen Projektidee?
  % Was wird unser Projekt am Ende voraussichtlich beinhalten?
\end{frame}


% --- Sektion 2: Progress on your project idea - Where are you currently at? What comes next? 
\section{Progress}

\begin{frame}{Progress | Project Requirements}
  \begin{alertblock}{Minimal Requirements (MVP)}
    \begin{itemize}
      \item Lauffähige AR App (Android)
      \item Anzeige eines Flut-Levels (visuell als Plane)
      \item Interface zur Anzeige der Wasserstandshöhe (als Textobjekt)
      \item Mock-API (selbes Format wie Real API) für Testing und Demonstration
    \end{itemize}
  \end{alertblock}
\end{frame}

\begin{frame}{Progress | Project Requirements | Current State}
  \begin{alertblock}{Minimal Requirements (MVP)}
    \begin{itemize}
      \item[\cmark] Lauffähige AR App (Android)
      \item[\cmark] Anzeige eines Flut-Levels (visuell als Plane)
      \item[\cmark] Interface zur Anzeige der Wasserstandshöhe (als Textobjekt)
      \item[\cmark] Mock-API (selbes Format wie Real API) für Testing und Demonstration
    \end{itemize}
  \end{alertblock}
\end{frame}

\begin{frame}{Implementation Status}
  \textbf{$\rightarrow$ Wir sind bisher im Zeitplan}
  \begin{block}{Aktueller Fortschritt und zukünftiger Fokus}
    \begin{itemize}
      \item[\textcolor{green}{\cmark}] MVP ist implementiert
      \item[\textcolor{orange}{\ding{109}}] Übergang von Mock-API zu echter API
      \item[\textcolor{orange}{\ding{109}}] Verbesserung der Grafik und UI Elemente
      \item[\textcolor{orange}{\ding{109}}] Dokumentation    
    \end{itemize}
  \end{block}
\end{frame}


\begin{frame}{Progress | Timeline}

    \includegraphics[height=0.73\textheight,keepaspectratio]{./img/progress.png}

\end{frame}

\section{Implementation Status}

\begin{frame}{Implementation Status | Screenshots}
  \begin{columns}[c]

    % Hochkant-Bild links
    \begin{column}{0.48\textwidth}
      \centering
      \includegraphics[height=0.75\textheight,keepaspectratio]{./img/screenshot_aktuellesProjekt.png}

      \vspace{0.4em}
      {\footnotesize Aktueller Projektstand}
    \end{column}

    % Querformat-Bild rechts
    \begin{column}{0.48\textwidth}
      \centering
      \includegraphics[width=\textwidth,keepaspectratio]{./img/screenshot_unity.png}

      \vspace{0.4em}
      {\footnotesize Unity-Ansicht}
    \end{column}

  \end{columns}
\end{frame}


% \begin{frame}{Implementation Status | Screenshots}
%   \begin{columns}[c]

%     \begin{column}{0.48\textwidth}
%        \centering
%       \framebox{\parbox{0.9\textwidth}{\centering
%           \vspace{2.5cm}
%           \textbf{Screenshot: AR APP} \\
%           \small\textit{Screenshot vom aktuellen Stand der App}
%           \vspace{2.5cm}
%       }}
%     \end{column}

%     \begin{column}{0.48\textwidth}
%       \centering
%       \framebox{\parbox{0.9\textwidth}{\centering
%           \vspace{2.5cm}
%           \textbf{Screenshot: Unity} \\
%           \small\textit{Screenshot von Unity in welchem das Projekt von der App geöffnet ist}
%           \vspace{2.5cm}
%       }}
%     \end{column}

%   \end{columns}\end{frame}

% ToDo: Wollen wir eines der Videos zeigen?

  \begin{frame}{Reflection}
    \begin{itemize}
      \item Läuft bisher gut
      \item Mock-API gute Entscheidung, um schnell zu testen
      \item Unity ist möglicherweise zu aufwendig für Anwendungsfall
    \end{itemize}
  \end{frame}

\begin{frame}{Reflection | Future Learnings}
    \begin{itemize}
      \item[\faPlus] Was macht eine intuitive und simple UI aus?
      \item[\faPlus] AR Darstellung von Wetterdaten einfacher zu verstehen?
    \end{itemize}
  \end{frame}



\begin{frame}{Organisation}
\begin{itemize}
    \item Treffen Montags
    \item Textchat
    \item GitHub
    \item Unity Cloud
\end{itemize}
\end{frame}

\begin{frame}
    \Huge{\centerline{\textbf{Fragen?}}}
\end{frame}






%----------------------------------------------------------------------------------------

\end{document}

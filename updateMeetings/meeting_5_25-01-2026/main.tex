%----------------------------------------------------------------------------------------
%    PACKAGES AND THEMES
%----------------------------------------------------------------------------------------

\documentclass[aspectratio=169,xcolor=dvipsnames]{beamer}
\usetheme{SimplePlus}

\usepackage{hyperref}
\usepackage{graphicx} % Allows including images
\usepackage{booktabs} % Allows the use of \toprule, \midrule and \bottomrule in tables
%\usepackage{lmodern}
\usepackage{pifont} % Für Checkmarks
\usepackage{minted} % For code snippets

% Definition von Checkmarks und Crossmarks
\newcommand{\cmark}{\textcolor{green!60!black}{\ding{51}}}
\newcommand{\xmark}{\textcolor{red}{\ding{55}}}
\newcommand{\omark}{\textcolor{orange}{\ding{110}}} % Offen/In Arbeit

%----------------------------------------------------------------------------------------
%    TITLE PAGE
%----------------------------------------------------------------------------------------

\title{AR Flood Hazard Maps}
\subtitle{Update Meeting 5}

\author{Frederik Alpers, Lea Plümacher, Marvin Hagemeister}

\institute
{
    Freie Universität Berlin % Your institution for the title page
}
\date{25.01.2026} % Date, can be changed to a custom date

%----------------------------------------------------------------------------------------
%    PRESENTATION SLIDES
%----------------------------------------------------------------------------------------

\begin{document}

\begin{frame}
    % Print the title page as the first slide
    \titlepage
\end{frame}

\begin{frame}{Overview}
    \tableofcontents
\end{frame}

\section{Aktueller Sprint}

\begin{frame}{Aktueller Sprint}
    \begin{block}{Ziele diese Woche}
        \begin{itemize}
            \item Angleichen der beiden Scenes (Demo und normal Mode)
            \item Debug Informationen ausblenden per Button
            \item\textbf{Bug-Fixing:} Location auf allen Geräten nutzbar machen
        \end{itemize}
    \end{block}
\end{frame}

\begin{frame}{Aktueller Sprint}
    \begin{block}{Erreichte Ziele}
        \begin{itemize}
            \item[\cmark] Debug Informationen ausblenden per Button
            \item[\cmark] \textbf{Bug-Fixing:} Location auf allen Geräten nutzbar machen
            \item[\cmark] Struktur für Dokumentation angefertigt
            \item[\omark] Verbessern der UI
        \end{itemize}
    \end{block}
\end{frame}

\section{Ergebnisse}

\begin{frame}{Debug Informationen ausblenden}
  \begin{figure}
    \centering
    \begin{minipage}{0.3\textwidth}
        \includegraphics[width=\linewidth]{img/screenshot_ar_flood_app_with_debug_info.jpg}
        \caption{Mit Debug Informationen}
    \end{minipage}\hfill
    \begin{minipage}{0.3\textwidth}
        \includegraphics[width=\linewidth]{img/screenshot_ar_flood_app_without_debug_info.jpg.jpg}
        \caption{Ohne Debug Informationen}
    \end{minipage}
  \end{figure}
\end{frame}

\begin{frame}[fragile]{Dokumentation - Code}
    \begin{itemize}
        \item Code Dokumentation verbessert
        \item Beispiel aus \texttt{API\_WaterLevel.cs}:
    \end{itemize}
\begin{minted}[fontsize=\scriptsize,linenos,frame=lines,breaklines]{csharp}
/// <summary>
/// Handles fetching and processing water level data from PegelOnline API.
/// This script is responsible for finding the closest water level measurement station
/// based on the user's GPS location and then retrieving real-time water level data
/// from that station. It also calculates absolute water height and displays it on UI.
/// </summary>
public class API_WaterLevel : MonoBehaviour
\end{minted}
\end{frame}

\begin{frame}{Dokumentation - Wiki}
  \begin{columns}[c]
    \begin{column}{0.48\textwidth}
        \textbf{Aufbau der Doku}
        \begin{itemize}
            \item Readme auf Github als Einstiegspunkt
            \item Hosting via Github Pages
            \item Erstellung via MkDocs
        \end{itemize}
    \end{column}
    \begin{column}{0.48\textwidth}
      \centering
      \includegraphics[width=0.9\textwidth]{img/screenshot_documentation_mkdocs_frontpage.png}
    \end{column}
  \end{columns}
\end{frame}

\begin{frame}{Dokumentation - Wiki - Struktur}
  \begin{columns}[c]
    \begin{column}{0.48\textwidth}
        \textbf{Struktur der Doku}
        \begin{itemize}
            \item Motivation
            \item User Dokumentation (How to use)
            \item Entwickler Dokumentation
                \begin{itemize}
                    \item Wie wurde es gebaut
                    \item Architektur
                    \item Future Works
                \end{itemize}
        \end{itemize}
    \end{column}
    \begin{column}{0.48\textwidth}
      \centering
      \includegraphics[width=0.9\textwidth]{img/screenshot_documentation_mkdocs_frontpage.png}
    \end{column}
  \end{columns}
\end{frame}

\section{Ausblick}

\begin{frame}{Ausblick - Nächste Schritte}
    \begin{itemize}
        \item[\omark] Angleichen Demo- und Normal mode
        \item[\omark] UI/UX Verbesserungen
        \item[\omark] Bugfixing und Dokumentation
        \item[\omark] Erstellen der finalen Präsentation
    \end{itemize}
\end{frame}

\begin{frame}{Organisation}
\begin{itemize}
    \item Treffen Montags
    \item Textchat
    \item GitHub
    \item Unity Cloud
\end{itemize}
\end{frame}

\begin{frame}
    \Huge{\centerline{\textbf{Fragen?}}}
\end{frame}

%----------------------------------------------------------------------------------------

\end{document}

%----------------------------------------------------------------------------------------
%    PACKAGES AND THEMES
%----------------------------------------------------------------------------------------

\documentclass[aspectratio=169,xcolor=dvipsnames]{beamer}
\usetheme{SimplePlus}

\usepackage{hyperref}
\usepackage{graphicx} % Allows including images
\usepackage{booktabs} % Allows the use of \toprule, \midrule and \bottomrule in tables
%\usepackage{lmodern}
\usepackage{pifont} % Für Checkmarks

% Definition von Checkmarks und Crossmarks
\newcommand{\cmark}{\textcolor{green!60!black}{\ding{51}}}
\newcommand{\xmark}{\textcolor{red}{\ding{55}}}
\newcommand{\omark}{\textcolor{orange}{\ding{110}}} % Offen/In Arbeit

%----------------------------------------------------------------------------------------
%    TITLE PAGE
%----------------------------------------------------------------------------------------

\title{AR Flood Hazard Maps}
\subtitle{Update Meeting 3}

\author{Frederik Alpers, Lea Plümacher, Marvin Hagemeister}

\institute
{
    Freie Universität Berlin % Your institution for the title page
}
\date{\today} % Date, can be changed to a custom date

%----------------------------------------------------------------------------------------
%    PRESENTATION SLIDES
%----------------------------------------------------------------------------------------

\begin{document}

\begin{frame}
    % Print the title page as the first slide
    \titlepage
\end{frame}

\begin{frame}{Overview}
    % Throughout your presentation, if you choose to use \section{} and \subsection{} commands, these will automatically be printed on this slide as an overview of your presentation
    \tableofcontents
\end{frame}

% --- Sektion 1: Aktueller Sprint Status ---
\section{Sprint Status}

\begin{frame}{Sprint Ziele \& Status}
    Der Sprint bis zum 01.12 ist abgeschlossen
    \vspace{0.5cm}
    \begin{itemize}
        \item[\cmark] \textbf{Erste AR-Anwendung}
        \item[\cmark] \textbf{Deployment:} App auf Gerät geflasht
        \item[\cmark] \textbf{GPS:} Position (Längengrad, Breitengrad und Höhe) bestimmt
        \item[\cmark] \textbf{Simulation:} Fake-API für Wasserstände (Geplant bis Ende der Woche)
    \end{itemize}
\end{frame}


\begin{frame}{Fake API}
    \begin{columns}[c]  % c = center vertical alignment of columns
    
        % Linke Spalte (vertikal zentriert)
        \begin{column}{0.45\textwidth}
            \vfill
            \begin{center}
                \textbf{Umgesetzte Funktionalität:}
                \begin{itemize}
                    \item Fake-API im Format der echten API erstellt
                    \item Erfolgreicher API-Aufruf
                    \item Anzeige der Daten im Projekt
                \end{itemize}
            \end{center}
            \vfill
        \end{column}

        % Rechte Spalte (Bild oben, vollständig angezeigt)
        \begin{column}{0.55\textwidth}
            \centering
            \includegraphics[height=0.8\textheight,keepaspectratio]{./img/API_Integration.png}
        \end{column}

    \end{columns}
\end{frame}


\begin{frame}{Sprint – Ziele \& Status}
    Aktueller Sprintzeitraum: bis zum 15.12.
    \vspace{0.5cm}

    \begin{itemize}
        \item[\omark] \textbf{Ziel Pegelstandssimulation:} Entwicklung der Logik zur Modellierung von Wasserständen
        \item[\cmark] \textbf{API-Anbindung Altitude:} Einbindung der Open-Meteo-API zur Ermittlung der Höhenlage
        \item[\omark] \textbf{Höhenberechnung:} Bestimmung des Hochwasserrisikos basierend auf Standort- und Pegeldaten
        \item[\omark] \textbf{Plane-Visualisierung:} Darstellung der ermittelten Höhe über eine 3D-Plane
    \end{itemize}
\end{frame}

\begin{frame}{Open-Meteo API}
    \begin{columns}[c]  % c = center vertical alignment of columns
    
        % Linke Spalte (vertikal zentriert)
        \begin{column}{0.45\textwidth}
            \vfill
            \begin{center}
                \textbf{Umgesetzte Funktionalität:}
                \begin{itemize}
                    \item Erfolgreicher API-Aufruf 
                    \item Anzeige der Daten im Projekt
                \end{itemize}
            \end{center}
            \vfill
        \end{column}

        % Rechte Spalte (Bild oben, vollständig angezeigt)
        \begin{column}{0.55\textwidth}
            \centering
            \includegraphics[height=0.8\textheight,keepaspectratio]{./img/elevation_API.jpg}
        \end{column}

    \end{columns}
\end{frame}

\begin{frame}{Plane-Visualisierung}
    \begin{columns}[c]  % c = center vertical alignment of columns
    
        % Linke Spalte (vertikal zentriert)
        \begin{column}{0.45\textwidth}
            \vfill
            \begin{center}
                \textbf{Umgesetzte Funktionalität:}
                \begin{itemize}
                    \item Darstellung einer Plane in AR
                    \item Höhenkontrolle der Plane 
                \end{itemize}
                \textbf{noch offen:}
                \begin{itemize}
                    \item Verknüpfung der Plane-Höhe mit Wasserstandsdaten
                    \item Höhe korrekt berechnen
                \end{itemize}
            \end{center}
            \vfill
        \end{column}

        % Rechte Spalte (zwei Screenshots nebeneinander)
        \begin{column}{0.55\textwidth}
            \centering
            \begin{minipage}{\textwidth}
                \centering
                \includegraphics[width=0.48\textwidth,keepaspectratio]{./img/screenshot_plane.jpeg}
                \hfill
                \includegraphics[width=0.48\textwidth,keepaspectratio]{./img/screenshot_plane2.jpeg}
            \end{minipage}
        \end{column}

    \end{columns}
\end{frame}





% --- Sektion 2: Detaillierte Ergebnisse (Bild + Text) ---
%\section{Erreichte Meilensteine}

% \begin{frame}{Erster AR-Prototyp}
%     \begin{columns}[c] % T = Top aligned
%         \begin{column}{0.48\textwidth}
%             \textbf{Funktionalität:}
%             \begin{itemize}
%                 \item Example AR-Unity Projekt erstellt / erweitert
%                 \item Erkennung von Flächen funktioniert 
%                 \item Funktioniert auf unseren Testgeräten
%             \end{itemize}
%         \end{column}
        
%         \begin{column}{0.48\textwidth}
%             \centering
%             \includegraphics[width=\textwidth]{./img/unity_screenshot.jpg}
% %            \framebox{\parbox{0.9\textwidth}{\centering
% %                \vspace{2.5cm}
% %                \textbf{Screenshot: AR Prototyp} \\
% %                \small\textit{Screenshot von Unity mit AR-Example-Project}
% %                \vspace{2.5cm}
% %            }}
%         \end{column}
%     \end{columns}
% \end{frame}

% \begin{frame}{GPS Location \& Hight}
%     \begin{columns}[c]
%         \begin{column}{0.48\textwidth}
%             \textbf{Implementierung:}
%             \begin{itemize}
%                 \item Zugriff auf Android Location Services
%                 \item Anzeige von Breitengrad, Längengrad, Genauigkeit und Höhe
%             \end{itemize}
%             \vspace{0.5cm}
%             \textit{Grundlage für das Finden der richtigen Messtation und Berechnung des Wasserstandes.}
%         \end{column}
        
%         \begin{column}{0.48\textwidth}
%             \centering
%             \includegraphics[width=0.5\textwidth]{./img/prot_1_location.jpg}
% %            \framebox{\parbox{0.9\textwidth}{\centering
% %                \vspace{2.5cm}
% %                \textbf{Screenshot: GPS Daten} \\
% %                \small\textit{UI mit Koordinaten-Anzeige}
% %                \vspace{2.5cm}
% %            }}
%         \end{column}
%     \end{columns}
% \end{frame}

% --- Sektion 3: Restlicher Sprint ---
\section{Nächste Schritte im Sprint}

\begin{frame}{Planung bis Sprint-Ende (Nächste Woche)}
    Offene Punkte für den aktuellen Sprint:
    
    \begin{block}{To-Do: Präzise Höhenberechnung der Plane \& Integration der Wasserstandsdaten}
        Für die korrekte Visualisierung des Hochwasserpegels müssen zwei Punkte umgesetzt werden:
        \begin{itemize}
            \item Berechnung des Höhenoffsets zwischen Kamera und Boden, um die absolute Höhe der AR-Plane zu bestimmen.
            \item Einbindung und Verarbeitung der Wasserstandshöhe aus der API, um die Plane relativ zum aktuellen Pegel korrekt darzustellen.
        \end{itemize}
    \end{block}
    
    \textbf{Geplante Umsetzung:}
    \begin{itemize}
        \item Abruf der Wasserstandsdaten aus der API und Umrechnung auf dieselbe Höhenreferenz.
        \item Berechnung des Differenzwerts (Wasserhöhe – Elevation) um Überflutung zu erfahren
        \item Brechnung Offset um Daten korrekt in AR darzustellen
    \end{itemize}
\end{frame}


% % --- Sektion 4: Projekt Definitionen ---
% \section{Projekt Definitionen}

% \begin{frame}{Projekt-Anforderungen}
%     Wir haben die Anforderungen neu definiert, um den Fokus für das MVP zu wahren.
    
%     \begin{alertblock}{Minimal Requirements (MVP)}
%         \begin{itemize}
%             \item Lauffähige AR App (Android)
%             \item Anzeige eines Flut-Levels (visuell als Plane)
%             \item Interface zur Anzeige der Wasserstandshöhe (als Textobjekt)
%             \item Mock-API (selbes Format wie Real API) für Testing und Demonstration
%         \end{itemize}
%     \end{alertblock}
% \end{frame}



% % --- Sektion 5: Projektmanagement ---
% \section{Organisation}

% \begin{frame}{Projektmanagement \& Roadmap}
%     \begin{columns}[c]
%         \begin{column}{0.48\textwidth}
%             \textbf{Strukturierung:}
%             \begin{itemize}
%                 \item Alle Anforderungen wurden in GitHub Issues überführt.
%                 \item Roadmap wurde visualisiert, um Abhängigkeiten zu klären.
%             \end{itemize}
%         \end{column}
        
%         \begin{column}{0.48\textwidth}
%             \centering
%             \includegraphics[width=\textwidth]{./img/roadmap.png}
% %            \framebox{\parbox{0.9\textwidth}{\centering
% %                \vspace{1.5cm}
% %                \textbf{Screenshot: GitHub/Roadmap} \\
% %                \small\textit{Übersicht der Issues \& Zeitplan}
% %                \vspace{1.5cm}
% %            }}
%         \end{column}
%     \end{columns}
% \end{frame}


\begin{frame}{Projekt-Anforderungen}
    Der aktuelle Stand, in wie weit unsere minimalen Anforderungen erfüllt sind.
    
    \begin{alertblock}{Minimal Requirements (MVP)}
        \begin{itemize}
            \item[\cmark] Lauffähige AR App (Android)
            \item Anzeige eines Flut-Levels (visuell als Plane) \begin{itemize}
                \item Flutlevel kann korrekt angezeigt werden, sobal die Entfernung zum Boden korrekt berechnet wird.
            \end{itemize}
            \item[\cmark] Interface zur Anzeige der Wasserstandshöhe (als Textobjekt)
            \item[\cmark] Mock-API (selbes Format wie Real API) für Testing und Demonstration
        \end{itemize}
    \end{alertblock}
\end{frame}

\begin{frame}{Organisation}
\begin{itemize}
    \item Treffen Montags
    \item Textchat
    \item GitHub
    \item Unity Cloud
\end{itemize}
\end{frame}

\begin{frame}
    \Huge{\centerline{\textbf{Fragen?}}}
\end{frame}






%----------------------------------------------------------------------------------------

\end{document}

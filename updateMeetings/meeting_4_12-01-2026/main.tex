%----------------------------------------------------------------------------------------
%    PACKAGES AND THEMES
%----------------------------------------------------------------------------------------

\documentclass[aspectratio=169,xcolor=dvipsnames]{beamer}
\usetheme{SimplePlus}

\usepackage{hyperref}
\usepackage{graphicx} % Allows including images
\usepackage{booktabs} % Allows the use of \toprule, \midrule and \bottomrule in tables
%\usepackage{lmodern}
\usepackage{pifont} % Für Checkmarks
\usepackage{pifont} % Für Checkmarks

% Definition von Checkmarks und Crossmarks
\newcommand{\cmark}{\textcolor{green!60!black}{\ding{51}}}
\newcommand{\xmark}{\textcolor{red}{\ding{55}}}
\newcommand{\omark}{\textcolor{orange}{\ding{110}}} % Offen/In Arbeit

%----------------------------------------------------------------------------------------
%    TITLE PAGE
%----------------------------------------------------------------------------------------

\title{AR Flood Hazard Maps}
\subtitle{Update Meeting 4}

\author{Frederik Alpers, Lea Plümacher, Marvin Hagemeister}

\institute
{
    Freie Universität Berlin % Your institution for the title page
}
\date{\today} % Date, can be changed to a custom date

%----------------------------------------------------------------------------------------
%    PRESENTATION SLIDES
%----------------------------------------------------------------------------------------

\begin{document}

\begin{frame}
    % Print the title page as the first slide
    \titlepage
\end{frame}

\begin{frame}{Overview}
    % Throughout your presentation, if you choose to use \section{} and \subsection{} commands, these will automatically be printed on this slide as an overview of your presentation
    \tableofcontents
\end{frame}

\section{Recap 2025}

\begin{frame}{Letzter Status}
    Projektstand im letzten Jahr:
    \vspace{0.5cm}

    \begin{itemize}
        \item[\cmark] \textbf{API-Anbindung Altitude:} Einbindung der Open-Meteo-API zur Ermittlung der Höhenlage
        \item[\cmark] \textbf{Plane-Visualisierung:} Darstellung der ermittelten Höhe über eine 3D-Plane
        \item[\cmark] \textbf{Demo-API:} Verwendung unserer Demo-API

    \end{itemize}
\end{frame}


\section{Aktueller Stand}

\begin{frame}{Ziele im aktuellen Sprint}
    Gesetzte Ziele im letzten Jahr:
    \vspace{0.5cm}

    \begin{itemize}
        \item[\omark] \textbf{Finden der nächsten Pegelstation:} Mittels Location und API-Call
        \item[\omark] \textbf{Debug-Daten deaktivieren:} User kann mittels Button Debug-Daten anzeigen
        \item[\omark] \textbf{UI-Anpassungen:} Besser aussehende UI

    \end{itemize}
\end{frame}

\begin{frame}{Ziele im aktuellen Sprint}
    Aktueller Stand:
    \vspace{0.5cm}

    \begin{itemize}
        \item[\cmark] \textbf{Finden der nächsten Pegelstation:} Mittels Location und API-Call
        \item[\cmark] \textbf{Debug-Daten deaktivieren:} User kann mittels Button Debug-Daten anzeigen
        \item[\xmark] \textbf{UI-Anpassungen:} Besser aussehende UI

    \end{itemize}
\end{frame}

\begin{frame}{Real-API}
  \begin{columns}[c]
    \begin{column}{0.48\textwidth}
      \textbf{Folgendes funktioniert:}
      \begin{itemize}
        \item[\cmark] Fetchen aller Stationen
        \item[\cmark] Finden der uuid der nächsten Station
        \item[\xmark] Probleme mit Location Permissions
      \end{itemize}
    \end{column}
    
    \begin{column}{0.48\textwidth}
      \centering
      \includegraphics[width=0.4\textwidth]{./img/5413430523264176902.jpg}

      \vspace{0.5em}
      {\footnotesize
      Screenshot: Anzeige der uuid der nächsten Station.}
    \end{column}

  \end{columns}
\end{frame}

\begin{frame}{Real-API}
  \begin{columns}[c]
    \begin{column}{0.48\textwidth}
      \textbf{Folgendes funktioniert:}
      \begin{itemize}
        \item[\cmark] Fetchen aller Stationen
        \item[\cmark] Finden der uuid der nächsten Station
        \item[\xmark] Probleme mit Location Permissions
      \end{itemize}
    \end{column}
    
    \begin{column}{0.48\textwidth}
      \centering
      \includegraphics[width=0.9\textwidth]{./img/5413430523264176902_cut.jpg}

      \vspace{0.5em}
      {\footnotesize
      Screenshot: Anzeige der uuid der nächsten Station.}
    \end{column}

  \end{columns}
\end{frame}


\begin{frame}{Wechsel Demo und Normal-Mode}
  \begin{columns}[c]
    \begin{column}{0.48\textwidth}
      \textbf{Folgendes funktioniert:}
      \begin{itemize}
        \item[\cmark] Button zum Wechsel zwischen zwei Szenen
        \item[\xmark] Scenen müssen noch angeglichen werden
      \end{itemize}
    \end{column}
    
    \begin{column}{0.48\textwidth}
      \centering
      \includegraphics[width=0.9\textwidth]{./img/5413430523264176902_cut.jpg}
      {\footnotesize
      Screenshot: Button zum Wechsel zwischen Demo- und Normalem Mode.}
    \end{column}

  \end{columns}
\end{frame}

\section{Ausblick}

\begin{frame}{Ausblock}
    Ziele bis zum nächsten Treffen:
    \vspace{0.5cm}

    \begin{itemize}
        \item[\omark] \textbf{Demo- und Nomal Mode:} Angleichen der UI oder Wechsel auf Ausblenden
        \item[\omark] \textbf{Bug-Fixing:} Location auf allen Geräten nutzbar machen
        \item[\omark] \textbf{Verbesserung der UI:} Besser aussehende UI und HDRI Map

    \end{itemize}
\end{frame}


\begin{frame}{Organisation}
\begin{itemize}
    \item Treffen Montags
    \item Textchat
    \item GitHub
    \item Unity Cloud
\end{itemize}
\end{frame}

\begin{frame}
    \Huge{\centerline{\textbf{Fragen?}}}
\end{frame}






%----------------------------------------------------------------------------------------

\end{document}
